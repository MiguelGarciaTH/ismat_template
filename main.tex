\documentclass[
 paper=A4,               % paper size --> A4 is default in Germany
    twoside=true,           % onesite or twoside printing
    openright,              % doublepage cleaning ends up right side
    parskip=full,           % spacing value / method for paragraphs
    chapterprefix=true,     % prefix for chapter marks
    12pt,                   % font size
    headings=normal,        % size of headings
    bibliography=totoc,     % include bib in toc
    listof=totoc,           % include listof entries in toc
    titlepage=on,           % own page for each title page
    captions=tableabove,    % display table captions above the float env
    draft=false,            % value for draft version
]{scrreprt}

%\usepackage[utf8x]{inputenc}
\usepackage[utf8]{inputenc}
\usepackage[T1]{fontenc}
\usepackage{newstyle}
\usepackage{tabularx}
\usepackage{amsmath,amsfonts,amssymb,amsthm,url}
\def\UrlBreaks{\do\/\do-}
\usepackage[portuguese,english]{babel}
\usepackage{times}
\usepackage{setspace}
\usepackage[sort&compress,numbers]{natbib}
\usepackage[vlined,ruled,commentsnumbered,linesnumbered]{algorithm2e} 
\usepackage{algorithmic}  
\usepackage{graphicx,multirow}
\usepackage{array}
\usepackage{multicol}
\usepackage{mathptmx} % use Times in math mode
%\usepackage{moreverb}
\usepackage{lscape,rotating}
\usepackage{hyperref}
\usepackage{fancyhdr}
\usepackage{lastpage}
\usepackage{nameref}
\usepackage{listings}
\usepackage{subfig}
\usepackage{tikz}
\usepackage{xcolor}
\usepackage{makeidx}
\usepackage{epsfig}
\usepackage{xspace}
%\usepackage{subfigure}
%\usepackage[show]{chato-notes}
%\usepackage{listings}
\usepackage{color}
\usepackage{marginnote}
\usepackage{flushend}
\usepackage{breakurl}
\usepackage{lipsum}  
\usepackage{longtable}
\usepackage[acronym]{glossaries}

%\makeindex


\theoremstyle{definition}
\newtheorem{defn}{Definition}[]

\newcommand*\circled[1]{\tikz[baseline=(char.base)]{
            \node[shape=circle,fill,inner sep=1pt] (char) {\textcolor{white}{#1}};}}

\fancyhf{} %
\lhead{\nouppercase {\leftmark}} %
\rhead{\nouppercase {\bf \thepage}}
\renewcommand{\headrulewidth}{0.1pt}

% Comando para inserir pagina em branco (inserida na numeracao, mas sem
% numero impresso) para quando e' preciso obrigar um capitulo a comecar
% do lado direito (pagina impar)
\newcommand{\LIMPA}{
\newpage
\mbox{}
\thispagestyle{empty}
}

% Igual, mas insere pagina com numero impresso (normalmente nao se usa)
\newcommand{\LIMPAC}{
\newpage
\mbox{}
\thispagestyle{plain}
}

%%%%%%%%%%%%%%%%%%%%%%%%%%%%%%%%%%%%%%%%%%%%%%%%%%%%%%%%%%%%%%%%%%%%%%%%%
% 				ALTERAR AQUI AS INFORMACOES RELATIVAS AO PROJECTO							%
%%%%%%%%%%%%%%%%%%%%%%%%%%%%%%%%%%%%%%%%%%%%%%%%%%%%%%%%%%%%%%%%%%%%%%%%%
\newcommand{\TITULO}{TITULO DO TRABALHO}
\newcommand{\Autor}{NOME DO ALUNO}
\newcommand{\AutorNumAluno}{NUMERO DO ALUNO}
\newcommand{\MIC}{Relatório de Metodologias de Investigação Científica}
\newcommand{\PFC}{Relatório de Trabalho Final de Curso}

%Orientador e CoOrientador *sem* titulos (e.g. Prof. Doutor)
\newcommand{\Orientador}{NOME ORIENTADOR}
\newcommand{\CoOrientador}{NOME CO-ORIENTADOR} %se nao se aplicar, nao importa o que aqui esteja

%Se aplicavel, o supervisor pode ter um titulo (Dr., Eng.) colocado aqui
\newcommand{\SupervisorInstituicao}{Nome Completo do Supervisor}  %se nao se aplicar, nao importa o que aqui esteja

\newcommand{\AnoLectivo}{2019/2020}
\newcommand{\Ano}{\Large{2020}}

% Comentar/descomentar conforme conveniente
\newcommand{\CADEIRA}{\MIC}

% Comentar/descomentar conforme conveniente
\newcommand{\IdiomaTese}{\selectlanguage{portuguese}}
%\newcommand{\IdiomaTese}{\selectlanguage{english}}
%%%%%%%%%%%%%%%%%%%%%%%%%%%%%%%%%%%%%%%%%%%%%%%%%%%%%%%%%%%%%%%%%%%%%%%%%%

\newcommand{\Cabecalho}{
\vspace{-0.5cm}\normalfont\normalfont
\vfill
\textsc{\normalsize\uppercase{Instituo Superior Manuel Teixeira Gomes}}\\
\vspace{1.5cm}
\includegraphics[scale=.9]{pic/logotipo_ISMAT.pdf}\\
}


\title{\TITULO}
\author{\Autor}
%\date{\today}

\makeglossaries

\begin{document}
\selectlanguage{portuguese}
\pagestyle{empty}

%%% Primeira capa
\begin{center}

\Cabecalho

\vspace{2.0cm}
\vfill
\IdiomaTese
\Large{\textbf{\TITULO}}\\
\vspace{1cm}
\vfill

\large{\textbf{\CADEIRA}}\\
\vspace{1.0cm}
\vfill
\Large{\textbf{\Autor}}\\
\vspace{1,8 cm}
\vfill
\large{Trabalho orientado por:}\\
\large{Prof. \Orientador} \\
% DESCOMENTAR a linha relevante (se alguma), removendo o % no inicio
e pelo Prof. \CoOrientador \\
\vspace{1 cm}
\vfill

\vspace{0.5cm}
\vfill
\vspace{-1cm}
\Ano
\end{center}


%%% Segunda capa
\begin{center}
\newpage
\thispagestyle{empty}
\mbox{}
\newpage
\cleardoublepage

\Cabecalho

\vspace{2.0cm}
\vfill
\IdiomaTese
\Large{\textbf{\TITULO}}\\
\vspace{1cm}
\vfill

\large{\textbf{\CADEIRA}}\\
\vspace{1.0cm}
\vfill
\Large{\textbf{\Autor}}\\
\vspace{1,8 cm}
\vfill
\large{Trabalho orientado por:}\\
\large{Prof. \Orientador} \\
% DESCOMENTAR a linha relevante (se alguma), removendo o % no inicio
e pelo Prof. \CoOrientador \\
\vspace{1 cm}
\vfill

\vspace{0.5cm}
\vfill
\vspace{-1cm}
\Ano
\end{center}
\newpage
\thispagestyle{empty}
\mbox{}
\newpage
\cleardoublepage


% Fim da capa
% ----------------------------------------------------------------------

\pagenumbering{roman}			% roman page numbing (invisible for empty page style)
\setcounter{page}{1}			% set page counter
\pagestyle{empty}

\selectlanguage{portuguese}

\pagestyle{plain}				% display just page numbers
% ----------------------------------------------------------------------
% Página do resumo em Inglês:
% 300 palavras
\selectlanguage{portuguese}
%\vspace*{2cm}
%\begin{center}
%\Large \bf Abstract
%\end{center}
%\vspace*{1cm} \setlength{\baselineskip}{0.6cm}
\chapter*{\TITULO{}  Declaração de autoria}
\label{chapter:copyright}
%\vspace*{-10mm}

Declaro ser o (a) autor(a) do trabalho apresentado neste relatório, sendo original e inédito. Autores e trabalhos consultados estão devidamente citados no texto e constam da listagem de referências incluída.
\vspace*{15mm}
%\begin{flushright}

\begin{center}
\underline{\hspace{4.5cm}}\\
(Assinatura)
\end{center}
%\end{flushright}
\vspace*{20mm}


%Na mesma folha deverá ainda aparecer a indicação de “Copyright”, seguida do seguinte texto:
O Instituto Superior Manuel Teixeira Gomes tem o direito, perpétuo e sem limites geográficos, de arquivar e publicitar este trabalho através de quaisquer meios, de o divulgar através de repositórios científicos e de admitir a sua cópia e distribuição com objetivos não comerciais, desde que seja dado crédito ao autor.
\vspace*{15mm}
\begin{center}
\underline{\hspace{4.5cm}}\\
(Assinatura)
\end{center}
%\LIMPA

% Fim da página do resumo em Inglês.
% ----------------------------------------------------------------------

\thispagestyle{empty}

\pagestyle{plain}				% display just page numbers
%\pagestyle{plain}

\vspace*{2cm}
\selectlanguage{portuguese}
\chapter*{Agradecimentos}

\lipsum[1-3]


\selectlanguage{english}


\LIMPA


\pagestyle{plain}				% display just page numbers
% ----------------------------------------------------------------------
% P�gina do resumo em Ingl�s:
% 300 palavras
\selectlanguage{portuguese}
%\vspace*{2cm}
%\begin{center}
%\Large \bf Abstract
%\end{center}
%\vspace*{1cm} \setlength{\baselineskip}{0.6cm}
\chapter*{Resumo}
\label{chapter:abstract_pt}
\vspace*{-10mm}
\lipsum[1-3]

\vfill

\begin{flushleft}
\textbf{Palavras-chave:}
Keyword1, Keyword1, Keyword3, Keyword4, Keyword5.
\end{flushleft}

%\LIMPA

% Fim da p�gina do resumo em Ingl�s.
% ----------------------------------------------------------------------


\pagestyle{plain}				% display just page numbers
\include{chapters/abstract/resumo_en}
%\cleardoublepage

%Lista de capitulos
\setcounter{secnumdepth}{3}
\setcounter{tocdepth}{2}		% define depth of toc
\tableofcontents				% display table of contents
%\addcontentsline {toc} {chapter} {Content}
\newpage
\thispagestyle{empty}
\mbox{}
\newpage
\cleardoublepage

%%Lista de figuras
\listoffigures
%\addcontentsline {toc} {chapter} {List of Figures}
\newpage
\thispagestyle{empty}
\mbox{}
\newpage
\cleardoublepage

%Lista de tabelas
\listoftables
%\addcontentsline {toc} {chapter} {List of Tables}
\newpage
%\thispagestyle{empty} 
%\mbox{}
%\newpage
%\cleardoublepage

%Lista de acronimos
\renewcommand{\glossaryname}{Acrónimos}
\include{acronimos}
\printglossaries
\newpage
\thispagestyle{empty}
\mbox{}
\newpage
\cleardoublepage

% --------------------------
% Body matter
% --------------------------

% ----------------------------------------------------------------------
% Inicio conteudo
\pagestyle{fancy}
\cleardoublepage

\selectlanguage{portuguese}
\pagenumbering{arabic}			% arabic page numbering
\setcounter{page}{1}			% set page counter
\pagestyle{maincontentstyle} 		% fancy header and footer

%%% ADICIONAR CAPITULOS
\include{chapters/introduction/introduction}
%%% \include{chapters/introduction/CAPITULO}
%%% \include{chapters/introduction/CAPITULO}

% Fim do conteudo
% ----------------------------------------------------------------------

% Glossario

\LIMPA
\bibliographystyle{IEEEtran}
\bibliography{chapters/references/references}
%\addcontentsline {toc} {chapter} {Bibliography}

\end{document}

